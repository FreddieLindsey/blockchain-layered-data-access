\section{Introduction}

\begin{displayquote}{
  "\textbf{When it comes to control over our own data, health data must be where we draw the line.}"~\parencite{wilbankstopol:2016:article}
}\end{displayquote}

Over the last 350 years, the general public has proceeded, often unwittingly, to give up their right to privacy by exchanging personal data for the convenience of the modern world. One might attribute the origin of this movement in the UK to the introduction of paper money by the Bank of England~\parencite{bankofengland:2016:online} in 1694. This event heralded the idea of giving information about a person to an institution, be it a corporation or a government, in exchange for convenience. Before this time, one might have kept their savings 'under the mattress' and therefore there would be no sharing of one's wealth with another party. Whilst bank notes were introduced as a means to raise funds for a war, they also required the depositors (as a whole) to identify exactly how much money they had as a group. At this stage, this imposes no constraint on the depositor to give up any part of his unique identity, only form part of a wider, anonymous identity (the group).
