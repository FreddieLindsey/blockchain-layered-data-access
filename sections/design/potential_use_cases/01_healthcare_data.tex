\subsubsection{Healthcare Data}

The notion of completely decentralised healthcare data is certainly an attractive one.

% Private data, public network
Firstly, from a security standpoint, the onus is not on any one hospital, doctor's surgery, or sub-contractor to uphold the security of a network. Since the network is public by design, all block data transferred through the network needs to be encrypted. The requirement for encryption means that the network security, in terms of data leakage, would never be upheld by a human, but by the cryptography underpinning the encryption scheme chosen.

% Decentralisation
Secondly, without the existence of a compromisable, centralised data lake, data (which mostly will exist across multiple nodes in the network) must be tampered with or removed from every location where it is stored to in fact remove or compromise it's existence in the network. On evaluation of recent global attacks, including to UK NHS hospital trusts, it is evident that providing better security of our data should always remain a high priority. Whilst encrypting our data does nothing to stop a malicious attacker from tampering with data, decentralising data such that copies are maintained in multiple locations, and ensuring that these locations are write-once prevents the tampering or compromise of existing copies of data.

% Decentralisation
Thirdly, when a patient relocates, changes their GP surgery, or moves from military to civilian health care, their previous healthcare provider is required to transfer all their data to the new provider. Should this be within the same organisation this can be problematic. However, in the scenario when a patient moves country, or when the organisation responsible for their health care changes, the transfer of data becomes much more difficult. Furthermore, should a patient wish to evaluate their health records, or share and discuss their health records with any other party, this is not possible currently. Decentralising access and control of health records is a possible solution to these problems.

Further more, for health care professionals who might be required to visit patients, the opportunity to have a local copy of data which they might need, fully cached and secured, could allow advances in treatment that are otherwise impossible whilst the requirement for internet access holds.
