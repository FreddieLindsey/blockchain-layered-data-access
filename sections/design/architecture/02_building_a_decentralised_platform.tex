\subsubsection{Building A Decentralised Platform}

The criteria established by the user design and the specific requirements and philosophies of the architecture yields the specific technology stack discussed below. This stack provides the backbone upon which the implementation presented is based.

\paragraph{Blockchain Technology}

As per the architecture criteria, the correct blockchain technology to use for this project is one compatible with an already developed and established public network. The four considered networks were \href{https://bitcoin.org/en/}{Bitcoin Blockchain}, \href{https://www.ethereum.org/}{Ethereum} (Ether), \href{https://www.hyperledger.org/}{Hyperledger}, and \href{https://www.corda.net/}{R3 Corda}. Whilst the blockchain technology being used for the project remains as only a proof of concept, the exact technology and it's shortcomings are not completely relevant. For this project, the exact cost of transacting with a particular chain will not be considered, with the ease of use, ability to integrate, and security proving to be the priority.

Bitcoin Blockchain, whilst the first widespread use of a public decentralised ledger, is designed as a cryptocurrency network foremost. This means there are strict limitations on the size of data that is contained by a transaction and the scripting language used has requirements to guarantee security during transfer of funds. It is not a blockchain designed with an application layer that is accessible to developers. This does not mean it is not possible to create applications on the Bitcoin Blockchain, but there are more suitable alternatives we can consider.

R3 Corda and Hyperledger both offer approaches engineered towards the demands of business and sharing application data globally. They both state they restrict the viewing of transactions to only those whom the transaction concerns. They both also have development environments available to allow building applications on their respective networks. 

With this said, Ethereum remains the standout choice for an application of this nature. Whilst Ethereum maintains the global transaction history as per the Bitcoin Blockchain, the development environment and ease of creating an application layer built atop the service provided a solution that most closely met the criteria of this project at this stage. The availability of \href{https://github.com/trufflesuite/truffle}{Truffle}~\footnote{As a pre-cursor to this project I fixed issues with a current boilerplate using Truffle and created my own boilerplate \href{https://github.com/FreddieLindsey/truffle-webpack-boilerplate}{repository}, available to the public to aid in future DApp creation.} (amongst other Ethereum development frameworks) and \href{https://github.com/ethereum/solidity}{Solidity} make creating contracts and deploying them relatively easy. Furthermore the availability of a local testnet~\footnote{EthereumJS offers \href{https://github.com/ethereumjs/testrpc}{testrpc} a JavaScript implementation of an Ethereum node which allows fast and extensible private, local node setup.} makes developing contracts in a secure environment possible.

\paragraph{Off-chain Storage}

Distributed (but centralised) storage, of the likes of Amazon Web Services S3 etc., is not suitable for this application. Whilst highly performant, a single corporation maintains the right to terminate the service at any juncture. This is not compatible with the objectives of this project.

Storj was considered as above, however this project is not intending to use a currently implemented system for private data sharing, but provide it's own solution to the data sharing problem. Therefore, Storj will not be used.

Whilst still in its infancy, a JavaScript implementation of an IPFS client \href{https://github.com/ipfs/js-ipfs}{is available} and has basic functionality allowing the putting and getting of objects onto and from the IPFS network.

\paragraph{Client Interface}

\paragraph{Architecture Overview}


