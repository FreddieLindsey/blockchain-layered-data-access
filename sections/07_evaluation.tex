\section{Evaluation Plan}

% Evaluation plan (1-2 pages). Project evaluation is very important, so it's important to think now about how you plan to measure success. For example,

% - what functionality do you need to demonstrate?
% - What experiments to you need to undertake and what outcome(s) would constitute success?
% - What benchmarks should you use?
% - How has your project extended the state of the art?
% - How do you measure qualitative aspects, such as ease of use?

% These are the sort of questions that your project evaluation should address; this section should outline your plan.

\subsection{Demonstrable core functionality}

Below are lists of functionalities that are required for the project to have achieved success. In the case where the user is expected to be able to give multiple inputs in an either-or fashion, partial success is still achieved by implementing a subset of those inputs.

\subsubsection{Secure Distributed Storage}

\begin{outline}
  \1 Store encrypted data only, such that readable by the primary data owner only
  \1 Data input and output should never be decrypted - this should be provable
\end{outline}

\subsubsection{Secure Layered Access}

\begin{outline}
  \1 Allow the use of different access layers across a dataset
  \1 A party $P_a$ that is a member of a data-layer access group $D_b$, but may have extra (superceding) permissions will have access that is an extension of a party $P_b$ who is only a member $D_b$.
  \1 Disallow a party to see data exists if they do not have read access
\end{outline}

\subsubsection{Time-based Access}

\begin{outline}
  \1 Allow any user to request data from any other user who they can identify
  \1 Allow nominated 3rd parties to access data upon the successful granting of a request
  \1 Granted access to data is time-dependent using one of two inputs:
    \2 Set remaining time period
    \2 Set access termination date
\end{outline}

\subsubsection{Transparent and Public Logging}

\begin{outline}
  \1 For every access of a file (through the system), a record is written to a public ledger
  \1 All records of user access must be encrypted such that the primary data owner is the only party that can read them
  \1 The collapse of the system would not stop a user from viewing the logs for their data
  \1 The use of a public ledger does not cost the primary data owner anything
\end{outline}

\subsubsection{Secure Access and Access Management}

\begin{outline}
  \1 Unauthorised access to a user's account (maliciously or otherwise) does not allow reading a user's data
  \1 An actor must not be able to write to the access system such that they gain unauthorised access to data
\end{outline}

\subsection{Experimentation and Validation}

In order to verify that the above functionalities have been met, a series of experiments will need to be performed. These will include but are not limited to:

\begin{outline}
  \1 Create two users. Use the first to request data from the second (given a username or other identity parameter).
  \1 Simulate the use of a security hierachy and observe whether the system is able to handle this as one would expect.
  \1 Validate that data for which access has been granted is accessible with the correct access permissions (multiple tests required)
  \1 Validate that data for which access is given in a time-sensitive manner is no longer available once this time period expires
  \1 Attempt to write transactions to the ledger the application uses to gain access to the user data
  \1 Ensure that under single-user and multi-user loads, access is correctly implemented
  \1 Simulate a malicious attack on a user's account and attempt to retrieve their data. Record what user security information is required as a minimum to access any part of the user's secure data.
\end{outline}

\subsection{User testing and evaluation}

Qualitative user data will be assessed using the front-end of the application created for demonstration purposes. It is important that users who would currently access and update such systems do not feel pain in using the developed prototype. It is also important that the user experience is similar to that expected by potential users. It is my intention to use members of the college community of varying technical abilities and select members of the public who work in relevant industries to test the useability of the application and give feedback to improve the user experience.

I will attend the Wearable Technology Show\footnote{London, UK-based event taking place on 7-8 March 2017 \url{http://www.wearabletechnologyshow.net/home}} where I will try to get as much market data as possible on the viability and demand for such an application in the market place. This will be largely from wearable technology providers who, for the health care case study, would likely provide the infrastructure for data input from end users.

% User feedback is necessary
% Comprehensive user feedback
% Concrete evidence of success: quantitative


% During the ideation stage of the project, and as part of my research to determine the criteria that would need to be met to deem the project a success, I quickly determined that a case study (or multiple) would be vital to give the project context. Multiple markets have been considered, but two seemed particularly appropriate:
%
% \begin{outline}
%   \1 Healthcare data
%   \1 National Security data
% \end{outline}
%
% As one of the largest markets for private data shared by a public company, I agreed with my supervisor that a suitable case study would be to provide secure accessible storage to front digital health data in the UK.
%
% Further to agreeing this, I have received input from an industry specialist, Dr. Robert Learney\footnote{Dr. Learney is a Dyson scholar from the Department of Bioengineering, Imperial College London}, in the field of digital health data. We discussed the current situation of digital health data in the UK market, with particular reference to the National Health Service~\footnote{\url{http://www.nhs.uk/pages/home.aspx}} (NHS) and what would be a requirement of a proof of concept that would solve some of the many issues with the current distribution and storage of current health data.
%
% The core criteria that this project must adhere to are:
%
% \begin{outline}
%   \1 Secure distributed storage
%   \1 Time-based access
%   \1 Transparent, public logging
%   \1 Secure access and access management
% \end{outline}
%
% Below, I have applied the use-case to the above criteria to outline the most significant output criteria to determine the project's success.
