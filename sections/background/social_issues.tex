% - Privatisation of data
%   - Google, Facebook, NHS, Banks, retail stores (loyalty programs)
%   - Data Protection Act (limitations and corporate-focus)
% - User choice (who do I want to have my data and how)
% - Online identities
%   - Global identity tracking
%   - Conglomerate identity providers
\subsection{Social Issues}

At the core of the motivation for this project lay several issues corresponding to the way in which society has been manipulated over time. It is my belief that we find ourselves in the current position without any ownership of our data because we've been keen (even greedy) as a society to reap the benefits of our data without considering the longer term security effects. We have dismissed the need to care and be responsible for our data. Below, I have highlighted the key domains in which we lack control that we should have over our personal data. Whilst written as a piece of fiction, we should be aware and concerned that ignoring the social issues with data transfer allows a world to form much similar to that of George Orwell's 1984~\autocite{orwell:1984:book} - we consider the likes of corporations synonymous with that of the 'Big Brother' character.

\subsubsection{Commoditisation of personal (and private) data}

There is no doubt that search tools such as those offered by Google and Microsoft, retail stores such as those offered by Amazon, and social networks such as Facebook and Twitter, dramatically enhance our lives and give us capabilities we would never have otherwise.

\subsubsection{Freedom to use personal data}
