\subsection{Off-chain storage}

Whilst the storage of data is well researched, since the aim of this project is to evaluate the applicability of new and immature software to solve key social and technological challenges, there is an opportunity to discover up and coming technology in the storage space.

When using blockchain as the central data lake for an application, we must consider where we store the data that it references. Whilst is is possible to store data on blockchain itself, it is impractical and extremely expensive as discussed below. Storing data on the chain, assuming a transaction will be accepted on Ethereum with a gas price of 0.2 microether, at the current prices~\footnote{\$244.96 / ETH as of 4th June 2017. Breakdown of transaction costs at \href{https://www.cryptocompare.com/coins/guides/what-is-the-gas-in-ethereum/}{Crypto Currency}}, correlates to a cost of 263,023.8 USD per GB.

$$
\begin{aligned}
5 * 1024^{3} * 0.2 * 10^{-6} &=& 1,073.74 \text{ ETH (2sf)} \\
&=& 263,023.80 \text{ USD (2sf)}
\end{aligned}
$$

Bear in mind that I have ignored other transaction costs, including submitting and storing execution cycles, and indeed a limit on block size exists on Ethereum~\footnote{\href{https://ethstats.net/}{EthStats} showed a gas limit of the order of 4.7 million as of 5th June 2017.}. Henceforth it is trivial that it would be inappropriate to store any data on blockchain other than that which represents state of the application.

To solve this issue, we therefore need to assume that data needs to be stored by some other manner such that it is readily available, but only addressed (not stored) in blockchain. Below are two solutions.

\subsubsection{Centralised data storage}

There are many incumbents in the data storage market offering centralised, redundant data platforms. The most popular at the time include \href{https://aws.amazon.com/s3/}{Amazon Web Services S3}, \href{https://cloud.google.com/storage/}{Google Cloud Storage}, and \href{https://azure.microsoft.com/en-gb/services/storage/}{Microsoft Azure Storage} to name a few. Whilst the specific storage prices for each service are not relevant, each provider offers the storage at a rate of the order of less than 0.1 USD / GB / month. Even if we imagine that the storage offered by Ethereum would remain persistent and accessible for fifty years, the cost of storage on modern centralised data storage platforms is of the order of more than four thousand times cheaper. 

It is not within the scope of this report to consider the performance benefits of different cloud providers and their individual architectures. None of the big players in the centralised data storage market offer similarly performant decentralised solutions that would be useful in the context of this project.

\subsubsection{Decentralised data storage}

Decentralised storage, using peer-to-peer networks for data flow, represents an interesting and novel approach for data storage. Until recently, the only commonly used peer-to-peer data storage was that of torrents. Some software providers~\footnote{Canonical provide images of their Ubuntu operating system available through torrent files. \href{https://www.ubuntu.com/download/alternative-downloads}{Ubuntu Alternative Downloads}} have made free software available through this medium before, although historically the use of torrents for mainstream content sharing has been limited to illegal activities largely involving the use of copyrighted material.

In more recent times, more modern implementations of peer-to-peer storage have arisen. The Inter-planetary File System~\footnote{IPFS is aiming to replace HTTP as the protocol we use to access data. \href{https://ipfs.io/}{\textit{IPFS is the distributed web}}} (IPFS) and Storj~\footnote{Storj aims to provide a decentralised and encrypted platform for sharing data. } are two options that somewhat extent the concept of peer-to-peer sharing through torrenting.

\paragraph{IPFS}

IPFS takes the peer-to-peer nature of torrenting and wraps it in a application and caching layers to provide a high-performance decentralised storage solution with the following benefits.

\begin{itemize}
	\item 
    	\textbf{Content-based addressing} \\
        All content sent to the IPFS network is addressed as part of a Merkle tree (tree of hashes) such that the integrity of the content can be validated by the hash, and that the hash (address) remains constant with the content.
    \item 
    	\textbf{Local storage of data (content)} \\
        An IPFS node caches hashes which are requested such that it can provide them more quickly to nearby peers including repeated requests by the owner of the node for the same hash.
    \item 
    	\textbf{Permanent Web} \\
        Due to the decentralised nature of IPFS, and the above content-based addressing, once a blob is present on the network, it cannot be revoked without all nodes which hold a copy deleting it. For content which has not yet been requested by any other node than the one where it was created, deletion might be possible, but it should be considered that IPFS represents somewhat permanent storage.
    \item
    	\textbf{Offline Web} \\
        IPFS provides the means for offline, localised websites and storage. Rather than requiring the backbone of the internet to provide connectivity between peers, local area networks of IPFS nodes are able to communicate and share data without barriers.
\end{itemize}

\paragraph{Storj}

Pronounced 'storage', Storj leverages a cryptocurrency backbone to provide a market for storage. There are two types of users, farmers and consumers. Farmers hold chunks of data on behalf of consumers for a monthly premium, with Storj managing the geolocation of that data to ensure it's security.

\subsubsection{Merkle DAG}

Whilst not directly related, as the underlying protocol behind IPFS, the 'Merkle DAG' is an interesting and useful concept in the context of file storage and partitioning.

Authored by \cite{merkle:1988:inbook} as a way to build trees where the content of a node could be validated as a function of other nodes, Merkle trees are an important 
