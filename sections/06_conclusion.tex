\section{Conclusion and Future Work}

This thesis set out to investigate the viability of a decentralised, secure, data-sharing platform. In doing so, multiple solutions have been considered and has resulted in a proof of concept being developed. One of the core objectives was to challenge the compromises users make by being actors in centralised data-sharing services, prolific globally. Through this lay the challenge of adapting new and up and coming technologies and progressing our understanding of their abilities.

This thesis shows that a secure, decentralised platform for data sharing can exist and affords its users considerable benefits over traditional centralised systems. This requires a single pair of keypairs (identity and encryption) and allows a completely secure system, whilst not imposing any more responsibility on the user than normal. Through the way the Web3 technologies interface with current browsers and computing systems, accessibility is maximised given the technology available.

It would be incorrect to say that the current implementation and work done is final. However, within the time contraints of the project, much work has been done to prove that a decentralised system can work and to discover some of the problems of designing and implementing such a system. Of most importance, the layered access system through group and individual key exchange. Secondly, this project represents the first public practical use of proxy re-encryption that I am aware of - other uses including commercial database products. Thirdly, the combination of decentralised compute and decentralised storage providing a system that is usable globally under different network conditions.

The system makes very strong progress with end-to-end encryption, decentralisation, a real-world use case, permissions, and group access. Accessibility could be improved, although not with currently implemented technologies.

Whilst there are areas of concern which can be solved in future works, this investigation has been successful with a strongly positive outcome.

\subsection{Future Work}

Having considered the progress of this project in meeting its objectives, I'd like to suggest some key areas where further work would achieve significant improvement.

\subsubsection{Further investigation of proxy re-encryption}

Proxy re-encryption has been used in this project to answer the question of whether truly decentralised privacy and data sharing can exist. In this way proxy re-encryption acts as a facilitator to the project. However, the current implementation faces security issues as evaluated. Further investigation of the suitability of proxy re-encryption for the project's objectives, and whether any other schemes might be more suitable, is necessary to determine whether the evaluated security issues can be overcome by modification, extension, or replacement of the scheme.

\subsubsection{ElGamal in the browser}

One of the greatest implementation issues is the dependency on having a Java runtime available to act as a local encryption proxy. Due to time constraints, this iteration of the project was not able to convert the Java-written cryptography library which implements the AFGH scheme into a JavaScript and Web compatible layer. Whilst the time and effort to implement this improvement would be very great, the value of making a secure JavaScript layer could enable cross-platform, cross-device use of the system and allow widespread availability.

\subsubsection{Formal monetisation}

Hypothetically a service which is designed to support the public interest shouldn't be a profitable entity, but one that re-invests in itself. However, if this project were commissioned as a decentralised service to support a public sector institution (such as the NHS), the participants in the network (those processing and validating transactions) would require some form of payment. I suggested in the evaluation that the transaction costs imposed by Ethereum, whilst irrelevant for the project's proof of concept, function as a means of regulating the service and incentivising honest (and timely) transaction processing.

I propose that investigation into the monetisation of the platform, particularly with a view to public sector use, would be of great benefit.

\subsubsection{Storage Blockchain}

Finally, we question whether any current blockchain is in fact suitable for this application, whether a single blockchain could run this application, and whether alternative proposals (such as multi and parachain use) might be more suitable.
